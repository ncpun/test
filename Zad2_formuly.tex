\documentclass[a4paper, 10pt] {article}
\usepackage{polski}
\usepackage{amsmath, amsfonts, amsthm}
\usepackage{amssymb,latexsym}
\usepackage[utf8]{inputenc}
\usepackage{graphicx}

\begin{document}

    \textmd{LABOLATORIUM 8}
    \\
    \\
    \\
    \underline{Przykład 1}
    \\
    Niech a i b będą dowolnymi liczbami dodatnimi. Korzystając z indukcji matematycznej wykazać, że
    \begin{equation}
         (a+b)^{n} < 2^{n} ( a^{n}+b^{n} )
    \end{equation}
\\
\\
\underline{Przykład 2 - cdots}
\\

Korzystając z zależności między średnimi wykazać, że dla dowolnej liczby naturalnej
n prawdziwa jest nierówność:$\frac{1}{n+2}+\frac{1}{n+3}+\frac{1}{n+4}+\cdots+\frac{1}{3n+4} \geq1 $
\\
\\
\underline{Przykład 3 - left,right}
Wykaż, że:
\\ a)$\left (^{30}_{29}  \right)+(^{31}_{29})=(^{32}_{30})-(^{5}_{5}) $
\\
b)
\begin{equation}
    \sum_{k=0}^{3n}(-1)^k(^{3n}_{k})=0
\end{equation}
 \\
\\
\underline{Przykład 4 - mathbb, array}Niech f : R → R będzie funkcją daną wzorem:
\begin{equation}
    f(x)=\left \{\begin{array}{ll} −x^2 - 6x - 8 & x\leq-3\\x + 2 & x>-3
    \end{arry}\right
\end{equation}
\\
\\
\underline {Przykład 5}
    \\
    \begin{center}
    $ \lim\limits_{n \to \infty}\frac{\sqrt[3]{5}+\sqrt{2}n^{3}}{3n^{2}+1}$
    \end{center}
 \\
\\
\underline{Przykład 6}
    \begin{center}
        $\sum\limits_{n=0}^{\infty} \frac{\sin^{2}n\pi}{\ln3n}$
    \end{center}
\\
\\
\underline{Przykład 7}
    \begin{center}
        $\int_{3}^{5} (x^{2}-5x+3)\cdot x dx$
    \end{center}
\\
\\
\underline{Przykład 8}
    \begin{center}
    $\forall \varepsilon>0  \exists n_0 \forall  
    n\geq n_0 |a_n-3|<\varepsilon$$
    \end{center} 
\\
\\
\underline{Przykład 9 - theorem}
 	\begin{theorem}\label{e} 
 	Każdy ciąg $\{x_n\}$ zbieżny w przestrzeni $(X,d)$ spełnia
 	\end{theorem}
\\
\\
\\
\underline{Przykład 15 - pmatrix}
    A_{m,n} = \begin{pmatrix}
    a_{1,1} & a_{1,2} & a_{1,3} & ... & a_{1,n}\\
    a_{2,1} & a_{2,2} & a_{2,3} & ... & a_{2,n}\\
    \vdots & \vdots & \vdots & \ddots & \vdots \\
    a_{m,1} & a_{m,2} & a_{m,3} & ... & a_{m,n}
\end{pmatrix}
    
dla dowolnego n\in\mathbb{N} 

\\
\\
\underline{Przykład 10 - tabular }
Utwórz tabelę (tekst w pierwszej kolumnie powinien być wy-
równany do lewej, natomiast w drugej i trzeciej wyśrodkowany)
	\begin{table}[!h]
	\centering
	\begin{tabular}{|l|c|c|}
	\hline
	% after \\: \hline or \cline{col1-col2} \cline{col3-col4} ...
	x & 0 & $\sqrt{3}$\\
	\hline
	f(x) & 3 & 5\\
	\hline
	\end{tabular}\\
	\caption{Tabela wartości funkcji}
	\label{tab:funkcja_f}
	\end{table}

\end{document}